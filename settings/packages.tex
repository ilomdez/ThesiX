% Packages used in ThesiX
%%-----------------------------------------------------------
% Embebing Fonts
% ¿Automatic in TexMaker? If you are using TeXStudio in Windows, Go to Options-> Configure TeXstudio->Commands->Ps2Pdf. In that field, just paste ps2pdf.exe -dPDFSETTINGS#/prepress -dEmbedAllFonts#true -dMaxSubsetPct#100 -dCompatibilityLevel#1.3 %.ps


%%-----------------------------------------------------------
% Debuging
%\usepackage{showframe}
%\usepackage{blindtext}
%\listfiles


%%------------------------------------------------------------
% Metadata
\usepackage[pdftex,hidelinks]{hyperref}
\hypersetup{pdftitle=\thetitle{}, pdfauthor={\thesisauthor}}
\hypersetup{pdfsubject={ThesiX LaTeX Template}}
\hypersetup{pdfkeywords={LaTeX, Thesis, Template, ILH}}


%%-----------------------------------------------------------
% General packages about encoding and appeareance
\usepackage[utf8]{inputenc}
\usepackage[top=1.75cm,bottom=1.75cm,left=2.5cm,right=2.5cm,includeheadfoot,asymmetric]{geometry}
\usepackage{newunicodechar}
\newunicodechar{µ}{\ensuremath{\mu}}
\newunicodechar{°}{º}
%\newunicodechar{α}{\ensuremath{\alpha}} %Esto se puede utilizar para caracteres chungos si va mal junto con el newunicodechar - fontec y inputtec de arriba , para los % quizas--
\usepackage{eurosym}
\usepackage{indentfirst}
\usepackage{enumitem}
\usepackage{amsmath}
\usepackage{amssymb} %Careful with the font loaded!! If it has already loaded the symmbols it may stop working
\usepackage{multirow}
%\usepackage{multicol}
\usepackage[shortcuts]{extdash} % unbreakable hyphen - dash
%\usepackage{xparse} %newcommand and newenvironment with multiple optional arguments
\usepackage{verbatim}
\usepackage{listings}
\usepackage{pdfpages}


%%------------------------------------------------------------
% Titles, chapters, headers, etc.
\usepackage{fancyhdr}
\usepackage{titlesec}
\titleformat{name=\chapter,numberless}
	{\filcenter\normalfont\huge\bfseries}{}{20pt}{\Huge}[\vspace{0.0ex}\titlerule]
%
\titleformat{name=\chapter}[display]
	{\Huge\rm\filcenter}
	{\titlerule[2pt]%
	\vspace{-3pt}%
	\titlerule
	\vspace{1pc}%
	\Huge\rm\bfseries\MakeUppercase{\chaptertitlename} \Huge\rm\bfseries\MakeUppercase{\thechapter}}
	{1pc}
	{\titlerule
	\vspace{1pc}%
	\Huge}
%
\titleformat*{\subsubsection}{\large\itshape} % Definition of subsubsection style
\setcounter{secnumdepth}{3} % Depth of sections numbering - 3 =subsubsections
\setcounter{tocdepth}{2} %Depth of table of content - 2 =subsections


%%------------------------------------------------------------
% REFERENCES & GLOSSARIES
\usepackage{bookmark} % allows to set bookmarks manually, useful for abstract etc
\usepackage{cleveref} % for subfigures references and more
\usepackage{nameref}  % to reference section names
\newcommand{\itnameref}[1]{\textit{\nameref{#1}}} % reference section name in italics
\newcommand{\quitnameref}[1]{``\textit{\nameref{#1}}''} % reference section name in italics and quotation mars
\usepackage{tocbibind}
\usepackage[toc, acronym, nopostdot, section=chapter, automake=immediate]{glossaries}  %sort=def hace que super no pueda separar por grupos de palabras
% sanitizesort=false hace que pase del \textit{•} y cosas de esas y organice directamente por lo que ve dentro aunque según he leido ralentiza bastante - guia page 15
\makeglossaries
\usepackage[style=alphabetic, backend=biber, sorting=nyt, defernumbers=true, maxbibnames=99]{biblatex}
\addbibresource{C:/Users/ilh/Documents/Bibtex/ThesisRef.bib}
%\usepackage[hyphenbreaks]{breakurl}
%\usepackage{csquotes}
%
\DefineBibliographyStrings{english}{%
  bibliography={References},
}
\defbibheading{secbib}[\bibname]{%
  \section*{#1}%
  \markboth{#1}{#1}
}
%
\defbibenvironment{bibliographyNUM}
  {\list
     {\printtext[labelnumberwidth]{%
        \printfield{prefixnumber}%
        \printfield{labelnumber}}}
     {\setlength{\labelwidth}{\labelnumberwidth}%
      \setlength{\leftmargin}{\labelwidth}%
      \setlength{\labelsep}{\biblabelsep}%
      \addtolength{\leftmargin}{\labelsep}%
      \setlength{\itemsep}{\bibitemsep}%
      \setlength{\parsep}{\bibparsep}}%
      \renewcommand*{\makelabel}[1]{\hss##1}}
  {\endlist}
  {\item}


%%------------------------------------------------------------
% IMAGES
\usepackage{float} % Useful for tables as well
\usepackage{graphicx}
\usepackage{caption}
\usepackage{subcaption}
\captionsetup{labelfont=bf,justification=justified,font=footnotesize}
%\usepackage{wrapfig}
\graphicspath{
{./images/}
{./images/Title/}
{./images/Cover/}
{./images/Intro/}
}
\usepackage{svg}
%\setsvg{inkscape={C:/"Program Files"/Inkscape/inkscape.exe -z -D}}
%\usepackage[inkscapeexe=C:/Program~Files/Inkscape/inscape.exe]{svg}
\svgpath{{./images/}{./images/cover/}{./images/chapter1/}{./images/chapter2/}{./images/chapter3/}{./images/chapter4/}{./images/chapter5/}{./images/title/}}
%\def\eattif#1.tif{#1}
%\DeclareGraphicsRule{.tif}{png}{.png}{`convert #1 \eattif#1-tif-converted-to.png }
\usepackage{epstopdf}
\epstopdfDeclareGraphicsRule{.tif}{png}{.png}{convert #1 \OutputFile}
\AppendGraphicsExtensions{.tif}


%%-------------------------------------------------------------
% TABLES
\usepackage{tabularx}
\usepackage{makecell} %group words
\usepackage{booktabs}
\usepackage{array}
\usepackage{arydshln} % dash lines
\renewcommand\tabularxcolumn[1]{m{#1}}% for vertical centering text in X column - Not working always somehow
\newcolumntype{Y}{>{\centering\arraybackslash}X} % same as X but horizontally allinged
\newcolumntype{M}{>{\centering\let\newline\\\arraybackslash\hspace{0pt}}m{.5cm}}
%\newcolumntype{N}[1]{>{\centering\let\newline\\\arraybackslash\hspace{0pt}}Y}
\newcolumntype{n}{>{\centering\arraybackslash}m} % same as m but horizontally allinged
%%------------------------------------------------------------
% Force Words Htphenation
\hyphenation{
Ga-In-N-As-Sb
}