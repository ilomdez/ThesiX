\section{Tables}\label{sec:Elements_tables}

\begin{lstlisting}
\begin{table}[ht]
	\centering
	\begin{tabular}{l | l | l}
	A & B & C \\
	\hline
	1 & 2 & 3 \\
	4 & 5 & 6
	\end{tabular}
	\caption{very basic table}
	\label{tab:example}
\end{table}
\end{lstlisting}

\begin{table}[ht]
	\centering
	\begin{tabular}{l | l | l}
	A & B & C \\
	\hline
	1 & 2 & 3 \\
	4 & 5 & 6
	\end{tabular}
	\caption{very basic table}
	\label{tab:example}
\end{table}
%
%\begin{table}[ht]
%    \begin{subtable}[h]{0.45\textwidth}
%        \centering
%        \begin{tabular}{l | l | l}
%        Day & Max Temp & Min Temp \\
%        \hline \hline
%        Mon & 20 & 13\\
%        Tue & 22 & 14\\
%        Wed & 23 & 12\\
%        Thurs & 25 & 13\\
%        Fri & 18 & 7\\
%        Sat & 15 & 13\\
%        Sun & 20 & 13
%        \end{tabular}
%        \caption{First Week}
%        \label{tab:week1}
%    \end{subtable}
%    \hfill
%    \begin{subtable}[h]{0.45\textwidth}
%        \centering
%        \begin{tabular}{l | c | c}
%        Day & Max Temp & Min Temp \\
%        \hline \hline
%        Mon & 17 & 11\\
%        Tue & 16 & 10\\
%        Wed & 14 & 8\\
%        Thurs & 12 & 5\\
%        Fri & 15 & 7\\
%        Sat & 16 & 12\\
%        Sun & 15 & 9
%        \end{tabular}
%        \caption{Second Week}
%        \label{tab:week2}
%    \end{subtable}
%    \caption{Max and min temps recorded in the first two weeks of July}
%    \label{tab:temps}
%\end{table}