\chapter{Introduction}\label{ch:Intro}
\begin{chapter_resume}
This chapter presents the thesis introduction.\\

The current scenario of the photovoltaic market is briefly reviewed, analysing the historical trends and future perspectives of the most important technologies.\\

This leads to the revision of the \soa{} of multijunction solar cells, on which this thesis focuses on. The three main approaches (new compound materials, metamorphic buffers and the wafer bonding technique) are analysed, summarizing their benefits and drawbacks.\\

The aforementioned revision set the basis to establish the thesis goals, which is devoted to enhancing the potential of multijunction solar cells by developing advanced architectures. One of the main pathways has been the traditional approach based on increasing the number of subcells, but other improvements such as the thinning of substrates or the use of graphene as a top contact have also been explored. Accordingly, a detailed explanation of the main objectives to be accomplished is provided.\\

Afterwards, the outline followed in the thesis to illustrate the work carried out is explained to ease the reader task of finding whatever required information.\\

Finally, the framework which has surrounded this thesis is described, accounting for the funding, the means provided by the group, colleagues or third party partners as well as the main achievements attained through this thesis.\\
\end{chapter_resume}

\input{./chapters/Intro_sections/AboutThesiX}
\section{Installation} % Dar más detalles
In order to use \ThesiX{} clone or download the whole repository at https://github.com/ilomdez/ThesiX. Compiling the code should be pretty straight forward, although some steps might vary depending on the operative system and the compiler chosen:

\subsection{Windows - TextMaker - MikTex}
This section has been tested under Windows 10, TexMaker 5.0.3 and MikTex 2.9.6972. The version of each package used in the compilation of the latest \ThesiX{} manual can be found in the log file in the main folder of the repository.

Follow the instructions below:
\begin{enumerate}
	\item Download the corresponding MikTex version (https://miktex.org/download) and install it.
	\item Download the corresponding TexMaker version (https://www.xm1math.net/texmaker/) and install it.
	\item Configure TexMaker as depicted in \fig\,\ref{fig:Intro_TexMakerConf}.

%%%%%%%%%%%%%%%%%%
\begin{figure}[h]
    \centering
    \includegraphics[width=.95\textwidth]{Intro_TexMakerConf.png}
    \caption[Screenshot of the main configuration settings in TexMaker.]{Screenshot of the main configuration settings in TexMaker.}
    \label{fig:Intro_TexMakerConf}    
\end{figure}
%%%%%%%%%%%%%%%%%%

	\item SVG images can be easily implemented although the explanation of how to install such a feature is still pending.
	\item TIF images can be easily implemented although the explanation of how to install such a feature is still pending.

\end{enumerate}


\subsection{Windows - TexStudio - MikTex}
Coming soon!

\subsection{Linux}
Coming soon!

\input{./chapters/Intro_sections/HowToUseThesiX}
\input{./chapters/Intro_sections/Outline}