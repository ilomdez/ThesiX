\section{Installation} % Dar más detalles
In order to use \ThesiX{} clone or download the whole repository at https://github.com/ilomdez/ThesiX. Compiling the code should be pretty straight forward, although some steps might vary depending on the operative system and the compiler chosen:

\subsection{Windows - TextMaker - MikTex}
This section has been tested under Windows 10, TexMaker 5.0.3 and MikTex 2.9.6972. The version of each package used in the compilation of the latest \ThesiX{} manual can be found in the log file in the main folder of the repository.

Follow the instructions below:
\begin{enumerate}
	\item Download the corresponding MikTex version (https://miktex.org/download) and install it.
	\item Download the corresponding TexMaker version (https://www.xm1math.net/texmaker/) and install it.
	\item Configure TexMaker as depicted in \fig\,\ref{fig:Intro_TexMakerConf}.

%%%%%%%%%%%%%%%%%%
\begin{figure}[h]
    \centering
    \includegraphics[width=.95\textwidth]{Intro_TexMakerConf.png}
    \caption[Screenshot of the main configuration settings in TexMaker.]{Screenshot of the main configuration settings in TexMaker.}
    \label{fig:Intro_TexMakerConf}    
\end{figure}
%%%%%%%%%%%%%%%%%%

	\item SVG images can be easily implemented although the explanation of how to install such a feature is still pending.
	\item TIF images can be easily implemented although the explanation of how to install such a feature is still pending.

\end{enumerate}


\subsection{Windows - TexStudio - MikTex}
Coming soon!

\subsection{Linux}
Coming soon!
